\documentclass[french]{article}

\usepackage[a4paper]{geometry}
\usepackage[utf8]{inputenc}
\usepackage[french]{babel}
\usepackage[T1]{fontenc}

%\usepackage{fullpage}
%\usepackage{graphicx}
\usepackage{listings}
\usepackage{xcolor}

% solarized pour la coloration du code!
\definecolor{base03}{HTML}{002B36}
\definecolor{base02}{HTML}{073642}
\definecolor{base01}{HTML}{586E75}
\definecolor{base00}{HTML}{657B83}
\definecolor{base0}{HTML}{839496}
\definecolor{base1}{HTML}{93A1A1}
\definecolor{base2}{HTML}{EEE8D5}
\definecolor{base3}{HTML}{FDF6E3}
\definecolor{yellow}{HTML}{B58900}
\definecolor{orange}{HTML}{CB4B16}
\definecolor{red}{HTML}{DC322F}
\definecolor{magenta}{HTML}{D33682}
\definecolor{violet}{HTML}{6C71C4}
\definecolor{blue}{HTML}{268BD2}
\definecolor{cyan}{HTML}{2AA198}
\definecolor{green}{HTML}{859900}

\lstset{
    basicstyle=\ttfamily,
    sensitive=true,
    backgroundcolor=\color{base3},
    keywordstyle=\color{cyan},
    commentstyle=\color{base1},
    stringstyle=\color{blue},
    numberstyle=\color{violet},
    breaklines=true,
    literate=
  {á}{{\'a}}1 {é}{{\'e}}1 {í}{{\'i}}1 {ó}{{\'o}}1 {ú}{{\'u}}1
  {Á}{{\'A}}1 {É}{{\'E}}1 {Í}{{\'I}}1 {Ó}{{\'O}}1 {Ú}{{\'U}}1
  {à}{{\`a}}1 {è}{{\`e}}1 {ì}{{\`i}}1 {ò}{{\`o}}1 {ù}{{\`u}}1
  {À}{{\`A}}1 {È}{{\'E}}1 {Ì}{{\`I}}1 {Ò}{{\`O}}1 {Ù}{{\`U}}1
  {ä}{{\"a}}1 {ë}{{\"e}}1 {ï}{{\"i}}1 {ö}{{\"o}}1 {ü}{{\"u}}1
  {Ä}{{\"A}}1 {Ë}{{\"E}}1 {Ï}{{\"I}}1 {Ö}{{\"O}}1 {Ü}{{\"U}}1
  {â}{{\^a}}1 {ê}{{\^e}}1 {î}{{\^i}}1 {ô}{{\^o}}1 {û}{{\^u}}1
  {Â}{{\^A}}1 {Ê}{{\^E}}1 {Î}{{\^I}}1 {Ô}{{\^O}}1 {Û}{{\^U}}1
  {œ}{{\oe}}1 {Œ}{{\OE}}1 {æ}{{\ae}}1 {Æ}{{\AE}}1 {ß}{{\ss}}1
  {ç}{{\c c}}1 {Ç}{{\c C}}1 {ø}{{\o}}1 {å}{{\r a}}1 {Å}{{\r A}}1
  {€}{{\EUR}}1 {£}{{\pounds}}1
}
\title{Devoir 2 \\
    Concept des langages de programmation}
\author{Vincent Antaki (p1038646)\\
		Émile Trottier (p1054384)}

% \renewcommand{\thesubsection}{\thesection.\alph{subsection}}

\begin{document}
	\maketitle
	
	\abstract
	    Implémentation d'un splay-tree en Scheme
    	
	\section{Fonctionnement général du programme}
		La fonction \textit{traiter} de notre programme a le comportement 
		typique d'une fonction main(). La compréhension de cette méthode
		donne une bonne idée générale de notre programme. Celle-ci est appelé 
		récursivement pour permettre autant de 
		requête que l'usager souhaite.\\
		
		La première étape est de reconnaître quel est le type de requête 
		l'usager demande. Pour cela, nous utilisons d'abord
		la fonction \textit{eval-expr} sur l'expression de l'usager qui nous 
		la met sous une bonne forme. Ainsi, si la requête contient un
		symbole '=' et qu'il n'y a aucun caractères après le 'key=', il s'agit d'une demande de suppression de mot et notre fonction
		la met sous la forme '(- key). Lors d'une entrée de la forme 
		'key=definition', la liste '(= key definition) est retournée parce 
		qu'il s'agit de l'ajout d'un mot au dictionnaire. Sinon, si le caractère '=' est tout simplement absent de la chaîne, c'est
		une recherche d'un mot et la forme '(\% key) est utilisée. Cette 
		manière de faire nous permet, une fois de retour dans la fonction 
		\textit{traiter} de différencier les cas d'utilisation et d'accéder 
		facilement la clé et la définition lorsque qu'applicable. \\
		 
			Ainsi, notre fonction \textit{traiter} n'a qu'à vérifier le 
			premier élément de la liste retournée par \textit{eval-expr} pour 
			savoir quel cas elle doit gérer. Dans le cas d'une suppression,
		nous appelons la fonction \textit{node-remove} qui l'exécute. Dans le 
		cas d'un ajout au dictionnaire par concaténation de terme, on 
		transforme la définition de sorte qu'elle soit une liste de caractères 
		(avec mémoire partagée bien sûr).
		Nous avons qu'à ajouter un nouveau noeud à l'arbre par la suite. Dans 
		le dernier cas où l'usager demande de rechercher un mot,
		celui-ci est recherché par la fonction \textit{node-find} de manière 
		récursive dans notre arbre.\\
			
			
			Par souci de simplicité et pour garder un niveau d’abstraction 
			assez élevé dans cette brève explication du programme,
		nous avons omis plusieurs détails, comme le traitement des erreurs, le 
		splay et la conservation des données. Il va sans dire qu’ils sont
		néanmoins pris en charge dans le programme, comme nous l’expliquerons dans les prochains paragraphes.		
		
	\section{Résolution de problème de programmation}
		\subsection{Analyse syntaxique d’une requête et lecture d’une requête 
		de longueur arbitraire}
			Tel que mentionné plus tôt, nous avons la fonction 
			\textit{eval-expr} qui nous transforme la requête sous une forme 
			que nous jugeons plus repésentative et concise. Si le test 
			d'appartenance au caractère '=' dans la requête réussi et que la 
			partie qui suit le '=' est vide, alors notre fonction reconnaît 
			une suppression et retourne une forme appropriée. Si la partie 
			après le '=' est non-vide, alors il s'agit
		de l'ajout d'un mot au dictionnaire. Deux situations sont alors possibles: soit la suite contient le caractère '+' et il faut faire
		une concaténation de définitions, soit il n'y apparaît pas et il 
		s'agit d'une définition bien normale. Si la clé du nouveau noeud 
		existe déjà dans le dictionnaire, la fonction \textit{node-insert} se 
		charge de remplacer le noeud existant. Sinon, c'est-à-dire si le signe 
		'=' n'apparait pas dans la requête, alors la conclusion de notre 
		programme est que l'usager veut rechercher la définition d'un certain 
		terme. Il est à noter que dès qu'une concaténation contient un seul 
		mot inexistant, l'opération est annulée et le message "terme inconnu" 
		est affiché et que peu importe
		la requête, son type est affiché à l'écran. \\
			
			En ce qui trait à la lecture de la demande de l'usager, la 
			fonction \textit{go} donnée s'en charge.\\
			
		Après chaque fin de requête de l'usager, c'est-à-dire à chaque fois qu'il tape la touche "enter", la ligne nous est passée sous forme
		de liste de caractère. Celle-ci peut donc être de manière 
		arbitrairement grande. Nous utilisons ensuite la fonction de test 
		\textit{member} ainsi que \textit{string-split} (notre implémentation 
		qui utilise des fonctions de type fold) pour partitionner 
		et modifier sa forme. \\
		
		Lorsqu'il s'agit de l'ajout d'un mot à définition
				simple, la définition du noeud est simplement la liste de 
				charactères après le signe '=' écrites par l'usager. 
				Néanmoins, la définition peut
				être une concaténation de définitions de mots déjà existants 
				dans l'arbre, ce qui demande un traitement supplémentaire. 
				Plusieurs fonctions 
				appelées l'une à la suite de l'autre permettent de trouver 
				toutes les définitions des mots de la concaténation et de 
				copier leur référence
				dans la définition du nouveau mot (ou du même mot s'il s'agit 
				en même 
				temps d'une redéfinition). De cette manière, nous n'avons pas 
				à nous 
				soucier 		de perdre une définition suite à une 
				concaténation 
				dans 
				une mauvaise situation, puisque le récupérateur automatique de 
				mémoire 
				intégré à Scheme
				(garbage collector en anglais) sait qu'une référence existe 
				dans un autre noeud de notre arbre si tel est le cas.
	
		\subsection{Représentation du dictionnaire et des définitions et 
		opération sur ces structures}
			Le langage Scheme est un langage de type fonctionnel; dans la 
			partie du programme que nous avons fait (i.e. pas la partie 
			fournie), nous utilisons un style fonctionnel pur car aucune
		des fonctions n'a d'effet de bord. 
			\\
			Nos noeuds sont représentés par une liste de quatre élément : 
			l'enfant de gauche, la clé, la définition et l'enfant de droite. 
			Les enfants sont soit nul, soit des noeuds. Notre dictionnaire 
			peut être représenté par le noeud à la racine puisque celui-ci 
			contient, à différente profondeur, tous les noeuds de l'arbre. 
			Avec la même logique, chaque noeud représente le sous-arbre dont 
			il est la racine. Il est à noter que ce dictionnaire est un 
			représentation valide d'un arbre binaire. Quant aux définitions et 
			clés, elles sont représentées par des listes de caractère. 
					Les toutes petites 
			fonctions suivantes nous permettent de mieux opérer sur les noeud 
			et sur leurs attributs.
			\\
		
			\lstinputlisting[firstline= 80,lastline=85,language=Lisp]{tp2.scm}
		
			 Pour le parcourir, on n'a qu'à aller dans le car ou le 
			cadddr du noeud
		associé à la racine. Il est à noter qu'un parcours infixe de l'arbre 
		correspond à un parcours en ordre alphabétique du dictionnaire.\\	
		
		Pour rajouter des noeuds, nous parcourons 
		récursivement ces deux sous structures jusqu'à trouver le sous-arbre 
		nul qu'il va
		remplacer pour devenir une feuille. Pour enlever des noeuds, nous 
		reconstruisons l'arbre au fur et à mesure que nous 
		recherchons le noeud. Une fois trouvé, nous retournons l'enfant du 
		noeud lorsqu'il n'en a qu'un seul ce qui aura comme effet que l'enfant 
		prendra la place du noeud supprimé. Dans le cas où il y a deux 
		enfants, nous retournons à la place du noeud supprimé le résultat de 
		l'insertion de l'enfant de droite dans l'enfant de gauche.\\
			
						Pour ce qui est du splay, la fonction récursive 
						\textit{node-splay} prend en entrée 
						la racine de l'arbre et la clé correspondante au noeud 
						à mettre à 
						la position de la racine. Cette fonction est appelée 
						après chaque 
						insertion et chaque recherche. \\
			
			
			
		\subsection{Affichage des réponses aux requêtes}
			La fonction non-modifiable go fournie par le professeur est responsable d'afficher les réponses de notre programme aux requêtes de l'usager. Pour ce faire, notre fonction principale traiter retourne une paire constituée du message à écrire à l'écran et du nouveau dictionnaire, possiblement changé suite à la requête. Par l'intermédaire de la fonction traiter-ligne, la paire est transmise à la fonction go qui lit ensuite tous les charactères du premier élément de la paire, comme on peut le constater dans le code de la fonction go:
			\\
			
			\lstinputlisting[firstline=383,lastline=384,language=Lisp]{tp2.scm}
			
			
			
			Par exemple, si l'usager demande à faire la recherche d'un mot et que celui-ci n'existe pas, le dictionnaire ne change pas et nous retournons la paire suivante à la fonction traiter-ligne:
			\\
			
			
			\lstinputlisting[firstline= 
			373,lastline=373,language=Lisp]{tp2.scm}
		
		\subsection{d) Traitement des erreurs}
			Nous n'avons vu aucun type de traitement d'erreur en classe et 
			n'en connaissions pas à la base pour le langage Scheme. Par 
			conséquent, nous avons tout simplement décidé d’utiliser beaucoup 
			de testage conditionnel pour différencier et traiter le plus de 
			cas possibles. Certaines fonctions traitent les erreurs et d'autre 
			retourne \#f ou la liste vide, tout dépendant de la nature de leur 
			opération. 
			Puisque nous pouvons prévoir toutes les erreurs qui peuvent 
			survenir selon les fonctions que nous utilisons, nous pouvons 
			tester chaque opération que nous faisons pour voir si une erreur 
			est survenue. Il faut néanmoins être très pointilleux et très bien 
			connaître tous les cas limites. Par exemple, la première chose que 
			nous faisons dans notre fonction traiter est de vérifier si 
			l'entrée de l'usager est vide, car dans ce cas aucun calcul n'est 
			nécessaire: \\
			
			
			\lstinputlisting[firstline= 
			346,lastline=349,language=Lisp]{tp2.scm}
			
		
		
		\section{Comparaison des langages C et Scheme}
				En ce qui a trait à l'analyse syntaxique des requêtes, aucun 
				des langages ne surpassent l'autre, les tests étant 
				sensiblement les mêmes. Pour ce qui est du typage, bien que 
				difficile à cerner au début, la représentation de tout en 
				liste par Scheme s'avère utile et facilite les opérations sur 
				les données. Néanmoins, le C avait l'avantage de nous 
				permettre de faire pas mal n'importe quelle opération quand on 
				le voulait, avec les effets de bord désiré. En effet, les 
				langages qui intègre la programmation impérative nous 
				permettent cela. Un langage de type fonctionnel, et même 
				fonctionnel pur 
				dans notre cas particulier, nous oblige a intégrer les 
				opérations dans des fonctions et de faire la composition de 
				fonction pour faire plusieurs opérations.
				\\
				Dans la même lignée, le C nous a simplifié la vie en nous 
				permettant de créer et de modifier des variables quand on le 
				voulait, où on le voulait. Comment ajouter un mot au 
				dictionnaire? Simplement en changeant directement les 
				pointeurs de l'enfant auquel il se rattache. En fait, peu 
				importe l'action à faire  dans le dictionnaire, le modifier ne 
				consiste qu'à modifier directement la variable qui le 
				contient. En Scheme, au contraire, il nous faut utiliser le 
				retour de fonctions pour le reconstruire, ce qui n'est pas 
				particulièrement plus compliqué mais inhabituel par rapport à 
				notre expérience de codage. En plus, ce style de codage nous 
				assure qu'aucun effet de bord ne sera 
				produit. Cette particularité des langages fonctionnels est à 
				double tranchant. D'une part, elle complique la vie au 
				programmer, surtout celui habitué à coder de manière 
				impérative depuis sa plus tendre enfance. De l'autre, elle 
				simplifie la compréhension du comportement des fonctions du 
				programme et améliore la réutilisation des fonctions.\\
				\\
				Comme vous vous en doutez, l'utilisation de fonctions est 
				primordial en programmation fonctionnelle. Le passage de 
				fonction comme paramètre à d'autres fonctions s'avère un atout 
				majeur que les langages comme C n'ont pas. Ceci permet 
				notamment l'utilisation de foldr et de foldl, deux fonctions 
				sensiblement pareilles qui appliquent récursivement des 
				fonctions sur les éléments d'une liste. Ainsi, pour applatir 
				une liste, en Scheme, nous utilisons la fonction suivante:
				\\
				
				
				\lstinputlisting[firstline=285,lastline=292,language=Lisp]{tp2.scm}
				
				
				En C, il aurait fallu faire une boucle, réfléchir aux bornes 
				de la boucles, etc.
				\\
				
				Sur le même ordre d'idée, le string-split des deux travaux 
				pratiques représentent bien l'avantage du foldr. Dans le 
				premier, nous avons du utiliser des tokens, dont l'utilisation 
				demandait un doctorat en la matière. Dans le deuxième travail, 
				nous avons tout simplement utiliser un foldr où la seule 
				petite complication était de penser à la fonction à lui passer 
				en paramètre, comme le montre le code suivant:
				\\
				
				
				\lstinputlisting[firstline=265,lastline=272,language=Lisp]{tp2.scm}
				
				
				
				On peut voir que notre programme C a été écrit en environ 350 
				lignes contre environ 400 pour notre programme Scheme. En 
				considérant qu'en C, nous n'avions pas implanté les 
				concaténations et qu'en Scheme, nous avons ajouté beaucoup de 
				test d'assertion (60 lignes) et nous avons implémenté le splay 
				(90 lignes avec les commentaires), on conclut que Scheme est 
				un langage plus concit et avec l'avantage indéniable d'avoir 
				une gestion de mémoire automatisée (difficulté majeure dans le 
				tp1).\\
				
				Parlons maintenant un peu des fonctions itératives de notre 
				programme. La fonction principale qui s'occupe du splay n'est 
				malheureusement pas itératives. Elle s'appelle récursivement, 
				mais ses appels récursifs ne sont pas en position terminal. En 
				fait, la majorité, sinon l'entièreté de nos fonctions 
				récursives ne sont pas sous formes itératives, parce que ce 
				n'est pas naturel de programmer ainsi et que le programme 
				fonctionne très 
				bien sans. Toutefois, nous utilisons très 
				souvent des foldr et des foldl dans nos fonctions 
				(non-récursives), et ceux-ci sont sous formes itératives. Tout 
				n'est pas noir!\\
				
				En bout de compte, nul n'est meilleur que l'autre, mais chaque 
				type de programmation a clairement des avantages et des 
				inconvénients. Certains sont plus adaptés que d'autre pour des 
				tâches particulière; Scheme l'est définitivement plus pour 
				l'implantation d'un arbre binaire. (Vincent préfère nettement 
				Scheme et grogne des injures sur la gestion de mémoire en C)
		
		
	
\end{document}
