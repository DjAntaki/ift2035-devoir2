\documentclass[french]{article}

\usepackage[a4paper]{geometry}
\usepackage[utf8]{inputenc}
\usepackage[french]{babel}
\usepackage[T1]{fontenc}

%\usepackage{fullpage}
%\usepackage{graphicx}
%\usepackage{listings}
%\usepackage{xcolor}

% solarized pour la coloration du code!
\definecolor{base03}{HTML}{002B36}
\definecolor{base02}{HTML}{073642}
\definecolor{base01}{HTML}{586E75}
\definecolor{base00}{HTML}{657B83}
\definecolor{base0}{HTML}{839496}
\definecolor{base1}{HTML}{93A1A1}
\definecolor{base2}{HTML}{EEE8D5}
\definecolor{base3}{HTML}{FDF6E3}
\definecolor{yellow}{HTML}{B58900}
\definecolor{orange}{HTML}{CB4B16}
\definecolor{red}{HTML}{DC322F}
\definecolor{magenta}{HTML}{D33682}
\definecolor{violet}{HTML}{6C71C4}
\definecolor{blue}{HTML}{268BD2}
\definecolor{cyan}{HTML}{2AA198}
\definecolor{green}{HTML}{859900}

\lstset{
    basicstyle=\ttfamily,
    sensitive=true,
    backgroundcolor=\color{base3},
    keywordstyle=\color{cyan},
    commentstyle=\color{base1},
    stringstyle=\color{blue},
    numberstyle=\color{violet},
    breaklines=true,
    literate=
  {á}{{\'a}}1 {é}{{\'e}}1 {í}{{\'i}}1 {ó}{{\'o}}1 {ú}{{\'u}}1
  {Á}{{\'A}}1 {É}{{\'E}}1 {Í}{{\'I}}1 {Ó}{{\'O}}1 {Ú}{{\'U}}1
  {à}{{\`a}}1 {è}{{\`e}}1 {ì}{{\`i}}1 {ò}{{\`o}}1 {ù}{{\`u}}1
  {À}{{\`A}}1 {È}{{\'E}}1 {Ì}{{\`I}}1 {Ò}{{\`O}}1 {Ù}{{\`U}}1
  {ä}{{\"a}}1 {ë}{{\"e}}1 {ï}{{\"i}}1 {ö}{{\"o}}1 {ü}{{\"u}}1
  {Ä}{{\"A}}1 {Ë}{{\"E}}1 {Ï}{{\"I}}1 {Ö}{{\"O}}1 {Ü}{{\"U}}1
  {â}{{\^a}}1 {ê}{{\^e}}1 {î}{{\^i}}1 {ô}{{\^o}}1 {û}{{\^u}}1
  {Â}{{\^A}}1 {Ê}{{\^E}}1 {Î}{{\^I}}1 {Ô}{{\^O}}1 {Û}{{\^U}}1
  {œ}{{\oe}}1 {Œ}{{\OE}}1 {æ}{{\ae}}1 {Æ}{{\AE}}1 {ß}{{\ss}}1
  {ç}{{\c c}}1 {Ç}{{\c C}}1 {ø}{{\o}}1 {å}{{\r a}}1 {Å}{{\r A}}1
  {€}{{\EUR}}1 {£}{{\pounds}}1
}

\title{Travail fabuleux de ?mile}
\author{?mile la magnifique}

\begin{document}
	\maketitle	
	
	\section{1. Fonctionnement général du programme}
		La fonction traiter de notre programme a le comportement typique d'une fonction main(). La compréhension de cette méthode
		donne une bonne idée générale de notre programme. Évidemment, celle-ci est appelé récursivement pour permettre autant de 
		requête que l'usager le veut.
			La première étape est de reconnaître quel est le type de requête l'usager demande. Pour cela, nous utilisons d'abord
		la fonction eval-expr sur l'expression de l'usager qui nous la met sous une bonne forme. Ainsi, si la requête contien un
		symbole '=' et qu'il n'y a aucun caractères après le 'key=', il s'agit d'une demande de suppression de mot et notre fonction
		la met sous la forme '(- key). Si la partie après le 'key=definition' est non-vide, la liste '(= key definition) est retournée parce 
		qu'il s'agit de l'ajout d'un mot au dictionnaire. Sinon, si le caractère '=' est tout simplement absent de la chaîne, c'est
		une recherche d'un mot et la forme '(key) est utilisée. Cette manière de faire nous permet de mieux séparer et diviser notre
		code. 
			Ainsi, notre fonction traiter n'a qu'à vérifier quelle forme d'expression elle doit gérer. Dans le cas d'une suppression,
		nous appelons une autre fonction qui l'exécute qui s'assure que le message retourné soit "suppression". Dans le cas d'un ajout 
		au dictionnaire, on transforme la définition de sorte qu'elle soit, même s'il sagit d'une concaténation, une liste de caractères.
		Nous avons qu'à ajouter un nouveau noeud à l'arbre par la suite. Dans le dernier cas ou l'usage demande de rechercher un mot,
		celui-ci est recherché par une autre fonction de manière récursive dans l'arbre binaire qui représente notre dictionnaire.
			Pour ce qui est du splay, la fonction node-splay prend en entrée la racine de l'arbre et la clé correspondante au noeud à mettre à la position de la racine. Cette fonction est appelée après chaque insertion et chaque recherche.
			Par souci de simplicité et pour garder un niveau d’abstraction assez élevé dans cette brève explication du programme,
		nous avons omis plusieurs détails, comme le traitement des erreurs et la conservation des données. Il va sans dire qu’ils sont
		néanmoins pris en charge dans le programme, comme nous l’expliquerons dans les prochains paragraphes.		
		
	\section{2. Résolution de problème de programmation}
		\subsection{a) Analyse syntaxique d’une requête et lecture d’une requête de longueur arbitraire}
			Comme je l'ai mentionné plus tôt, nous avons la fonction eval-expr qui nous transforme la requête sous une forme repésentative
		voulue. Si le test d'appartenance au caractère '=' dans la requête échoue et que la partie qui suit le '=' est vide, alors notre
		fonction reconnaît une suppression et re tourne une forme appropriée. Si la partie après le '=' est non-vide, alors il s'agit
		de l'ajout d'un mot au dictionnaire. Deux situations sont alors possibles: soit la suite contient le caractère '+' et il faut faire
		une concaténation, soit il n'y apparaît pas et il s'agit d'une définition bien normale. Sinon, c'est-à-dire si le signe '=' n'apparait
		pas dans la requête, alors la conclusion de notre programme est que l'usager veut recherche la définition d'un certain terme. Il est à noter
		que dès qu'une concaténation contient un seul mot inexistant, l'opération est annulée et le message "terme inconnu" est affiché et que peu importe
		la requête, son type est affiché à l'écran.
			En ce qui a trait à la lecture de la demande de l'usager, la fonction go donné par le professeur qu'on ne peut pas toucher s'en charge.
		Après chaque fin de requête de l'usager, c'est-à-dire à chaque fois qu'il tape la touche "enter", la ligne nous est passée sous forme
		de liste. Celle-ci peut donc être de manière arbitrairement grande. Nous utilisons ensuite la fonction de test member pour séparer 
		et modifier sa forme, et elle n'utilise pour traiter la liste ce qui garde la généralité du programme.
	
		\subsection{b) Représentation du dictionnaire et des définitions et opération sur ces structures}
			Le langage scheme est un langage de type fonctionnel; dans notre programme, il est même de type fonctionnel pur car aucune
		des fonctions n'a d'effet de bord. Il est aussi bon de savoir qu'il accepte les expressions en type préfixe, ce qui implique l'utilisation
		de parenthèses et donc de listes à tout bout de champ.
			Notre dictionnaire est donc une liste de listes imbriquées représenté sous forme d'arbre binaire. En fait, chaque noeud de l'arbre
		représente un mot du dictionnaire. Un noeuds est une liste constitué de quatre élément, dans cet ordre: le noeud enfant gauche,
		le terme, la définition et le noeud enfant droit. On peut aussi voir les enfants gauche et droit respectivement comme le sous-arbre
		gauche et droit du noeud, ou encore comme tous les mots qui viennent respectivement à la gauche puis à la droite du mot dans le dictionnaire. Les tout petites fonctions suivantes nous permettent de mieux opérer sur les noeud et sur leurs attributs.
			
			(define node-key cadr)
			(define node-definition caddr)
			(define node-lchild car)
			(define node-rchild cadddr)
			(define (node-create key definition lchild rchild) (list lchild key definition rchild))
			
			À la base, notre dictionnaire est donc la liste qui représente sa racine. Pour le parcourir, on n'a qu'à aller dans le car ou le cadddr du noeud
		associé à la racine. Pour rajouter des noeuds, nous parcourons récursivement ces deux sous structures jusqu'à trouvé le sous-arbre nul qu'il va
		remplacer pour devenir une feuille. Pour enlever des noeuds, opération plus délicate, nous remplaçons
			Quant aux définitions, elles n'y échappent pas: elles sont aussi représentées par des listes. Les tout petites fonctions suivantes nous permettent de mieux opérer sur les noeud et sur leurs attributs. Lorsqu'il s'agit de l'ajout d'un mot à définition
		simple, la définition du noeud est simplement la liste de charactères après le signe '=' écrites par l'usager. Néanmoins, la définition peut
		être une concaténation de définitions de mots déjà existants dans l'arbre, ce qui demande un traitement supplémentaire. Plusieurs fonctions 
		appelées l'une à la suite de l'autre permettent de trouver toutes les définitions des mots de la concaténation et de copier leur référence
		dans la définition du nouveau mot (ou du même mot s'il s'agit en même temps d'une redéfinition). De cette manière, nous n'avons pas à nous soucier
		de perdre une définition suite à une concaténation dans une mauvaise situation, puisque le récupérateur automatique de mémoire intégré à Scheme
		(garbage collector en anglais) sait qu'une référence existe dans un autre noeud de notre arbre si tel est le cas.
		
		\subsection{c) Affichage des réponses aux requêtes}
			La fonction non-modifiable go fournie par le professeur est responsable d'afficher les réponses de notre programme aux requêtes de l'usager. Pour ce faire, notre fonction principale traiter retourne une paire constituée du message à écrire à l'écran et du nouveau dictionnaire, possiblement changé suite à la requête. Par l'intermédaire de la fonction traiter-ligne, la paire est transmise à la fonction go qui lit ensuite tous les charactères du premier élément de la paire, comme on peut le constater dans le code de la fonction go:
			
			(let ((r (traiter-ligne ligne dict)))
            (for-each write-char (car r)))
			
			Par exemple, si l'usager demande à faire la recherche d'un mot et que celui-ci n'existe pas, le dictionnaire ne change pas et nous retournons la paire suivante à la fonction traiter-ligne:
			
			(cons (string->list "terme inconnu") dict)
		
		\subsection{d) Traitement des erreurs}
			Nous n'avons vu aucun type de traitement d'erreur en classe et n'en connaissions pas à la base pour le langage Scheme. Par conséquent, nous avons tout simplement décidé d’utiliser beaucoup de « if »  pour différencier et traiter le plus de cas possibles. Puisque nous pouvons prévoir toutes les erreurs qui peuvent survenir selon les fonctions que nous utilisons, nous pouvons tester chaque opération que nous faisons pour voir si une erreur est survenue. Il faut néanmoins être très pointilleux et très bien connaître tous les cas limites. Par exemple, la première chose que nous faisons dans notre fonction traiter est de vérifier si l'entrée de l'usager est vide, car dans ce cas aucun calcul n'est nécessaire:
			(define traiter
				(lambda (expr dict)
					(if (null? expr) (cons (string->list "entree vide") dict)
										  (faire le traitement de expr...)))
		
		\section{3. Comparaison des langages C et Scheme}
				En ce qui a trait à l'analyse syntaxique des requêtes, aucun des langages ne surpassent l'autre, les tests étant sensiblement les mêmes. Pour ce qui est du typage, bien que difficile à cerner au début, la représentation de tout en liste par Scheme s'avère utile et facilite les opérations sur les données. Néanmoins, le C avait l'avantage de nous permettre de faire pas mal n'importe quelle opération quand on le voulait. En effet, les langages impératifs nous permettent cela. Un langage de type fonctionnel, et même fonctionnel pur dans notre cas particulier, nous oblige a intégrer les opérations dans des fonctions et de faire la composition de fonction pour faire plusieurs opérations.
				Dans la même lignée, le C nous a simplifié la vie en nous permettant de créer et de modifier des variables quand on le voulait, où on le voulait. Comment ajouter un mot au dictionnaire? Simplement en changeant directement les pointeurs de l'enfant auquel il se rattache. En fait, peu importe l'action à faire  dans le dictionnaire, le modifier ne consiste qu'à modifier directement la variable qui le contient. En Scheme, au contraire, il nous faut utiliser le retour de fonctions pour le transformer, ce qui est plus compliqué mais ce qui assure qu'aucun effet de bord ne sera produit. Cette particularité des langages fonctionnels est à double tranchant. D'une part, elle complique la vie au programmer, surtout celui habitué à coder de manière impérative depuis sa plus tendre enfance. De l'autre, elle simplifie la compréhension du comportement des fonctions du programme et améliore la réutilisation des fonctions. Par exemple, une fonction à effet de bord pourrait changer la valeur d'une variable qui est utilisé à d'autres endroits dans le code et qui ne s'attend pas à être modifié là-bas.
				Comme vous l'aurez compris, l'utilisation de fonctions est primordial en programmation fonctionnelle. Le passage de fonction comme paramètre à d'autres fonctions s'avère un atout majeur que les langages comme C n'ont pas. Ceci permet notamment l'utilisation de foldr et de foldl, deux fonctions sensiblement pareilles qui appliquent récursivement des fonctions sur les éléments d'une liste. Ainsi, pour applatir une liste, en Scheme, nous utilisons la fonction suivante:
				(define (construire-def lst)
						(foldr
							(lambda(x y) (append x y))
							'()
							lst))
				En C, il aurait fallu faire une boucle, réfléchir aux bornes de la boucles, ...
				Sur le même ordre d'idée, le string-split des deux travaux pratiques représentent bien l'avantage du foldr. Dans le premier, nous avons du utiliser des tokens, dont l'utilisation demandait un doctorat en la matière. Dans le deuxième travail, nous avons tout simplement utiliser un foldr où la seule petite complication était de penser à la fonction à lui passer en paramètre, comme le montre le code suivant:
				
				(define string-split
					(lambda (str chr)
						(foldr  
							(lambda (x y)
								(if (equal? x chr) 
									(cons '() y)
									(cons (cons x (car y)) (cdr y))))
							'(()) str)))
				On peut voir que notre programme C a été écrit en environ 350 lignes contre environ 400 pour notre programme Scheme. En considérant qu'en C, nous n'avions pas implanté les concaténations et qu'en Scheme, nous avons ajouté beaucoup de test d'assertion (60 lignes) et nous avons implémenté le splay (90 lignes avec les commentaires), on conclut que Scheme est un langage plus concit et avec l'avantage indéniable d'avoir une gestion de mémoire automatisée (difficulté majeure dans le tp1).
				Parlons maintenant un peu des fonctions itératives de notre programme. La fonction principale qui s'occupe du splay n'est malheureusement pas itératives. Elle s'appelle récursivement, mais ses appels récursifs ne sont pas en position terminal. En fait, la majorité, sinon l'entièreté de nos fonctions récursives ne sont pas sous formes itératives, parce que ce n'est pas naturel de programmer ainsi et que le programme très bien sans faire cet effort. Toutefois, nous utilisons très souvent des foldr et des foldl dans nos fonctions (non-récursives), et ceux-ci sont sous formes itératives. Tout n'est pas noir!
				En bout de compte, nul n'est meilleur que l'autre, mais chaque type de programmation a clairement des avantages et des inconvénients.
		
		
	
\end{document}
